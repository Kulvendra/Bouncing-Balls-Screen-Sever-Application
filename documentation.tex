\documentclass[]{article}
\begin{document}

\title{Documentation of final submission for assingnment screen saver}
\author{Ayush Singh Pal(2016csj0018),
kulvendra singh(2016csj0026),
vetagiri hrushikesh(2016csj0018)}
\maketitle

\begin{abstract}
This document represents the objectives that we were able to achieve which were set during the assignment announcement.
It also give an insight of detailed instructions that are necessary to run the program and keys which ae being used to run various functions via keyboard.
\end{abstract}
\section{Content from read me}
 Screensaver
 
-The application displays the front view of 3-D screensaver in which various number of balls are bouncing on the screen following the laws of physics.Remember it will look like a 2-D display however it is indeed 3-D more specifically a front view of 3-D display this argument is shown by fact that they don't collide each time with each other and seems to pass by becauses at that time they are at some distance from each other and their depth are different.

-It allows user to enter any number of balls .

-user can use specified keys from keyboard to access features/functions  provided by program.

General Features

- The position, speed and colour of each ball is initially random. 
- balls do not overlap intially. 
- The ball-ball collision and ball-wall collision is treated elastic.

Features that  user can access using keys

-speed of the ball can be increased.
-screensaver can be paused anytime.
-screensaver can be resumed anytime.
-size of the ball can be increaed and decreased.

Inside the GUI window, N number of bouncing balls will be seen on the screen. 4 buttons are included for various uses.

-	 KEY 'P': Pressing this key once will pause the movement of all the balls. Pressing it again will resume their movement.

-	KEY 'Q': This button is used to increase the speed of all balls. 

-	KEY 'I': This button increases the radius of each ball.   

-	KEY 'J': This button decreases the radius of each ball. 

\section{Name of all files that are being used}

-ball.cpp

-ball.h

-display.cpp

-display.h

-main.cpp

-main.h

-physics.cpp

-physics.h
 
\section{Objectives completed}

-screen saver is 3-D one with laws of physics in ball-ball collision and ball-wall collison are being perfectly followed.

-user can enter any number of balls at the starting of program.

-user can access various functions such as incresing speed,increases/decreases the size ,play and pause the screen at any time with specified keys as mentioned.

\section{objectives omitted}

-moving camera

-addition of terrain


	


\end{document}